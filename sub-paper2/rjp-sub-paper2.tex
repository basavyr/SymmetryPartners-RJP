\documentclass[myclassdoc,debug]{rjparticle}
%use the following command when typesetting your paper:
%\documentclass{rjparticle}
\usepackage{graphicx}

\title{Extensive Study of the Positive and Negative Parity Wobbling States for an Odd-Mass Triaxial Nucleus Ii: Geometrical Interpretation} 

\author[1,2,$a$]{R. Poenaru}
\author[2,3,$b$]{A. A. Raduta}

\affil[1]{Doctoral School of Physics, University of Bucharest, Bucharest, Romania\\
\email[a]{robert.poenaru@drd.unibuc.ro} }
\affil[2]{Department of Theoretical Physics - \textit{Horia Hulubei} National Institute for Physics and Nuclear Engineering, M\u{a}gurele-Bucharest, Romania\\
\email[b]{raduta@nipne.ro} (corresponding author)}
\affil[3]{Academy of Romanian Scientists, Bucharest, Romania}

\keywords{Nuclear Structure, Triaxial Nuclei, Wobbling Motion, Parity Symmetry, Signature Partners, Strong Deformation}

\pacs{01.30.-y, 01.30.Ww, 01.30.Xx, 99.00.Bogus}

\hyphenation{rjp-ar-ti-cle}

%%%%%%%%%%%%%%%%%%%%%%%%%%%%%%%%%%%%%%%%%%%%%%%%%%%%%%%%%%%%%%%%%%%%%%%%%%%%%%%
%Please, do not remove the following lines!
%%%%%%%%%%%%%%%%%%%%%%%%%%%%%%%%%%%%%%%%%%%%%%%%%%%%%%%%%%%%%%%%%%%%%%%%%%%%%%%
%\RJPVolume{63}{2018}
%\RJPNumber{1-2}
%\RJPPages{}{}
%\columntitle{Wobbling Nucleus II}
%\date{}
%\dedication{}
%\domaintitle{}
%\keywords{}
%\pacs{01.30.-y, 01.30.Ww, 01.30.Xx, 99.00.Bogus}
%%%%%%%%%%%%%%%%%%%%%%%%%%%%%%%%%%%%%%%%%%%%%%%%%%%%%%%%%%%%%%%%%%%%%%%%%%%%%%%

\begin{document}
%%%%%%%%%%%%%%%%%%%%%%%%%%%%%%%%%%%%%%%%%%%%%%%%%%%%%%%%%%%%%%%%%%%%%%%%%%%%%%%
%Please, remove these lines when typesetting your document!
%%%%%%%%%%%%%%%%%%%%%%%%%%%%%%%%%%%%%%%%%%%%%%%%%%%%%%%%%%%%%%%%%%%%%%%%%%%%%%%
\lstset{%
basicstyle=\small,
language=[AlLaTeX]TeX,
columns=fullflexible,
%keepspaces=true,
showspaces=true,
showstringspaces=false,
keywordstyle=[2]\ttfamily,
identifierstyle=,
texcsstyle=*\ttfamily,
commentstyle=\color{gray},
string=[s]<>,
morestring=[b]',
stringstyle=\emph,
breaklines=true,
deletekeywords={list},
moretexcs={authnote,keywords,pacs},
}
%%%%%%%%%%%%%%%%%%%%%%%%%%%%%%%%%%%%%%%%%%%%%%%%%%%%%%%%%%%%%%%%%%%%%%%%%%%%%%%
\maketitle

\begin{abstract}
A new interpretation of the wobbling structure in $^{163}$Lu is developed. Four wobbling bands are experimentally known in this isotope, where three are wobbling phonon excitations $TSD_{2,3,4}$, and the ground state band, which is $TSD_1$. In this work, a particle-triaxial rotor coupling is considered in a product space of single-particle and collective core states. The single-particle states describe a $j=i_{13/2}$ proton, while the core states characterize the triaxial rotor and are either of positive parity, when the bands $TSD_{1,2,3}$ are concerned or of negative parity for the $TSD_4$ band. There are five free parameters, three moments of inertia, the strength of the particle-core interaction, and the $\gamma$ deformation. A very good description of all 62 experimental states is obtained, with a mean square error of about $80\ \text{keV}$. The newly obtained features evidenced in the present work enrich the knowledge about the wobbling properties of $^{163}$Lu.
\end{abstract}

\section{Introduction compatibility}
The Romanian Journal of Physics (RJP) style was designed to allow authors, who use mainly \LaTeX ~for typesetting their papers, to  submit contributions to this journal of the Romanian Academy Publishing House. 

%\begin{table}[h!t]%
%\caption{This table is taken from RJP volume \textbf{50}(1-2) from page 43 (2005). It gives the ``
%\textit{number of bound states dependence on the radius of space curvature for $\alpha = 0.005$, $U_0 = 1$}''.}
%\centering
%\begin{tabular}{|c|c|}
%\hline
%Value $\rho$ & Value $\varepsilon$ \cr
%\hline
%$\rho = 50$ & -- \cr
%$\rho = 100$ & -- \cr
%$\rho = 250$ & $\varepsilon_1 = 0.0289$ \cr
%$\rho = 400$ & $\varepsilon_1 = 0.3772$ \cr
%$\rho = 1000$ & $\varepsilon_1 = 0.4142$, $\varepsilon_2 = 0.8495$ \cr
%\hline
%\end{tabular}
%\label{table1}
%\end{table}


\begin{acknowledgement}
This work was supported by UEFISCU, through the project \textbf{PCE-16/2021}.
\end{acknowledgement}


\begin{thebibliography}{99}
\bibitem{knuth}D. E. Knuth, D. R. Bibby, ``\textit{The \TeX book}'', 20th edn. (AMS \& Addison-Wesley Publ. Co., 1991).
\bibitem{dknuthhp} D. E. Knuth homepage: \href{http://www-cs-faculty.stanford.edu/~knuth/}{\small\ttfamily www-cs-faculty.stanford.edu/\~{}knuth}.

\end{thebibliography}


\end{document}